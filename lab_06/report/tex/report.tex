\documentclass{bmstu}

\begin{document}

\makereporttitle
{Информатика и системы управления (ИУ)}
{Программное обеспечение ЭВМ и информационные технологии (ИУ7)}
{\textbf{6}}
{Моделирование}
{Моделирование работы системы массового обслуживания (GPSS)}
{}
{ИУ7-75Б}
{И. О. Артемьев}
{И. В. Рудаков}


\setcounter{page}{2}
\renewcommand{\contentsname}{Содержание} 
\tableofcontents

\chapter{Задание}

Для выполнения лабораторной необходимо смоделировать работу 
системы массового обслуживания, состоящую из генератора и 
обслуживающего аппарата. Генератор работает по равномерному 
закону распределения, а обслуживающий аппарат --- по закону распределения 
Пуассона. 
Необходимо определить максимальную длину очереди без потерь. 
Предусмотреть возможность возврата обработанной заявки обратно 
на вход обслуживающего аппарата (задается вероятностью). 
Реализовать на языке имитационного моделирования GPSS.

\chapter{Теоретическая часть}

\section{Равномерное распределение}

Функция равномерного распределения:

\begin{equation}
    F(x) =
    \begin{cases}
            0, x < a, \\
            \begin{aligned}
                \frac{x -  a}{b - a}, x \in [a, b], 
            \end{aligned}\\
            0, x > b. \\
    \end{cases}
\end{equation}

Функция плотности равномерного распределения:

\begin{equation}
    f(x) =
    \begin{cases}
            \begin{aligned}
                \frac{1}{b - a}, x \in [a, b], 
            \end{aligned}\\
            0, else. \\
    \end{cases}
\end{equation}

\section{Распределение Пуассона}

Биномиальное распределение с параметрами $n$ и $p$ -- это распределение количества <<успехов>> в последовательности из $n$ независимых случайных экспериментов, таких, что вероятность <<успеха>> в каждом из них постоянна и равна $p$.

Распределение Пуассона -- это частный случай биномиального распределения при $n \gg 0$ и $p \to 0$. Распределение Пуассона также называют законом <<редких>> событий, так как оно всегда проявляется там, где производится большое число испытаний, в каждом из которых с малой вероятностью происходит <<редкое>> событие.


Дискретная случайная величина $X$ имеет закон распределения Пуассона с параметром $\lambda$ ($X \sim \Pi(\lambda)$), где $\lambda > 0$, если она принимает значения $0, 1, 2,...$ с вероятностями:

\begin{equation}
	P(X = k)= e^{-\lambda}\frac{\lambda^{k}}{k!}, \quad k \in \{0, 1, 2, ...\}
\end{equation}


Параметр $\lambda$ распределения Пуассона -- это среднее количество успешных испытаний в заданной области возможных исходов. 


Соответствующая функция распределения принимает вид:

\begin{equation}
F(x) = P(X < x) = \sum_{k=0}^{x-1}P(X = k) = e^{-\lambda}\sum_{k=0}^{x-1}\frac{\lambda^{k}}{k!} 
\end{equation}

Для дискретной случайной величины не существует функции плотности распределения вероятностей. 

\chapter{Код программы}

\imgs{code}{h!}{0.35}{Код программы}

\chapter{Демонстрация работы программы}


\imgs{result}{h!}{0.35}{Результат работы программы}

\end{document}
