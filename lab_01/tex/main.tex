\documentclass{bmstu}

\begin{document}
\makereporttitle
{Информатика и системы управления (ИУ)}
{Программное обеспечение ЭВМ и информационные технологии (ИУ7)}
{\textbf{1}}
{Моделирование}
{Изучение функций распределения и функций плотности распределения}
{1}
{ИУ7-75Б}
{И.О. Артемьев}
{И.В. Рудаков}

\setcounter{page}{2}
\renewcommand{\contentsname}{СОДЕРЖАНИЕ} 
\tableofcontents

\chapter{Задание}
Разработать программу для построения графиков функции распределения и функции плотности распределения для следующих распределений: 
\begin{itemize}
	\item равномерное распределение;
	\item распределение Пуассона.
\end{itemize} 

Разработать интерфейс, предоставляющий возможность выбора закона распределения и указания его параметров.

\chapter{Теоретическая часть}

\section{Равномерное распределение}

Функция плотности распределения $f(x)$ случайной величины $X$, имеющей равномерное распределение на отрезке $[a, b]$ ($X \sim R(a, b)$), где $a, b \in R$, имеет следующий вид:
\begin{equation}
	f(x)=\begin{cases}
		\frac{1}{b - a}, & x \in [a, b] \\
		0, & \text{иначе}.
	\end{cases}
\end{equation}

Соответствующая функция распределения $F(x) = \int_{-\infty}^{x}f(t)dt$ принимает вид: 
\begin{equation}
	F(x)=\begin{cases}
		0, & x < a, \\
		\frac{x - a}{b - a}, & x \in [a, b] \\
		1, & x > b
	\end{cases}
\end{equation}


\section{Распределение Пуассона}

Биномиальное распределение с параметрами $n$ и $p$ -- это распределение количества <<успехов>> в последовательности из $n$ независимых случайных экспериментов, таких, что вероятность <<успеха>> в каждом из них постоянна и равна $p$.

Распределение Пуассона -- это частный случай биномиального распределения при $n \gg 0$ и $p \to 0$. Распределение Пуассона также называют законом <<редких>> событий, так как оно всегда проявляется там, где производится большое число испытаний, в каждом из которых с малой вероятностью происходит <<редкое>> событие.


Дискретная случайная величина $X$ имеет закон распределения Пуассона с параметром $\lambda$ ($X \sim \Pi(\lambda)$), где $\lambda > 0$, если она принимает значения $0, 1, 2,...$ с вероятностями:

\begin{equation}
	P(X = k)= e^{-\lambda}\frac{\lambda^{k}}{k!}, \quad k \in \{0, 1, 2, ...\}
\end{equation}


Параметр $\lambda$ распределения Пуассона -- это среднее количество успешных испытаний в заданной области возможных исходов. 


Соответствующая функция распределения принимает вид:

\begin{equation}
F(x) = P(X < x) = \sum_{k=0}^{x-1}P(X = k) = e^{-\lambda}\sum_{k=0}^{x-1}\frac{\lambda^{k}}{k!} 
\end{equation}

Для дискретной случайной величины не существует функции плотности распределения вероятностей. 


\chapter{Результаты работы программы}


\section{Равномерное распределение}

На рисунках \ref{img:ravn1} и \ref{img:ravn2} приведены результаты построения графиков функций плотности $f(x)$ и распределения $F(x)$ для случайных величин $X \sim R(-5.4, 5)$ и $X \sim R(1, 4)$, соответственно.

\imgs{ravn1}{h!}{0.55}{Графики функций плотности $f(x)$ и распределения $F(x)$ для случайной величины $X \sim R(-5.4, 5)$.}
\imgs{ravn2}{h!}{0.55}{Графики функций плотности $f(x)$ и распределения $F(x)$ для случайной величины $X \sim R(1, 4)$.}

% \imgs{erlangDistr}{h!}{0.33}{Распределение Эрланга}

\clearpage


\section{Распределение Пуассона}

На рисунках \ref{img:puas1} и \ref{img:puas2} приведены результаты построения графиков функции вероятности $P(x)$ и распределения $F(x)$ на отрезке $x \in [-10, 20]$ для случайных величин $X \sim \Pi(1)$, и $X \sim \Pi(5)$, соответственно.

\imgs{puas1}{h!}{0.55}{Графики функций вероятности $P(x)$ и распределения $F(x)$ для случайной величины $X \sim \Pi(1)$.}
\imgs{puas2}{h!}{0.55}{Графики функций вероятности $P(x)$ и распределения $F(x)$ для случайной величины $X \sim \Pi(5)$.}


\end{document}